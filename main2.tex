\documentclass{informs3}
\usepackage{apacite}
\usepackage[english]{babel}
\usepackage[utf8x]{inputenc}
\usepackage{amsmath}
\usepackage{basic} % refers to the style we're using, basic.sty
\usepackage[colorinlistoftodos]{todonotes}

\begin{document}

\title{Contrasting the Branding Effect of Cost Per View vs Cost Per Click Campaigns}
\author{Chad Crowe}

\maketitle

\begin{abstract}

  Existing works have predicted user behavior on social media using either image or text data. This research takes a further step with a methodology to combine image and text data for predicting user behavior on social media. The methodology is applied to 350k Facebook firm-generated content (FGC) posts. This research demonstrates that the proposed methodology that utilizes both text and image data for predicting user behavior outperforms text-only or image-only models for predicting user likes, shares, comments, and comment sentiment. The results validate the need for using both image and text data for predicting user behavior on social media.

\end{abstract}
  
\section{Introduction}

\subsection{Machine Learning on Social Media Data}

Existing research recognizes the value of modeling user behavior on social media using machine learning models \cite{Li2015, 8029313, Ohsawa2013, Liu2012, Li2015}. This pattern of research reflects the desire to understand consumer behavior on social media \cite{Fisher2009}. These studies have focused on particular metrics, including user click-through rates \cite{Li2015}, user interaction with Facebook posts \cite{8029313}, predicting user tendency to follow pages \cite{Ohsawa2013}, and user sentiment \cite{Liu2012,Wang2015}. A large amount of research is concerned with better understanding user behavior on social media.

The existing research understands user behaviors through behavior models, which often include machine learning models \cite{Li2015, 8029313, Ohsawa2013, Liu2012, Li2015}.  Examples include statistical models \cite{Li2015}, neural networks \cite{8029313}, text idf models \cite{Ohsawa2013}, opinion mining \cite{Liu2012}, and sentiment analysis of images \cite{Wang2015}. Each  example demonstrate the common methodology of modeling user behavior using machine learning models in order to better understand social analytics and user behavior. Modeling user behaviors can provide beneficial research insights about social analytics.

Convolutional Neural Networks (CNNs) perform well at working with images and are used in correspondence with social media images. They have been used for gender classification \cite{Hassner2015}, for visual sentiment \cite{Segalin2017, Xu2014}, to detect sarcasm on Twitter \cite{Poria2016}, for detecting stress in social media images \cite{Lin2014}, to perform social media profiling \cite{Segalin2017}, to predict social media popularity \cite{Gelli2015}, and to predict which posts will receive the post clicks \cite{Khosla2014}. CNNs are the standard in the realm of social media for image analysis \cite{Hassner2015}. 

Models trained on text data exist to understand user behavior. Facebook likes have been predicted for hospital data using only post text data \cite{8029313}. Research has used NLP to predict the likelihood a user will follow a Facebook page \cite{Ohsawa2013}. Other models have used text data for opinion mining \cite{Liu2012}. Text data is also frequently used for modeling user behavior on social media.

\subsection{Gap}

However, a gap exists in current research when modeling user behavior with social media data because researchers fail to incorporate multiple data types, despite the availability of both image and text data. An example is performing sentiment analysis of posts with images \cite{Wang2015} but failing to incorporate the post's text into the model. The same is true for using text-data to predict a post's CTR but ignoring its associated image data \cite{Li2015}. The gap in research is a failure to utilize multiple available data types when modeling user behavior on social media.

The gap is demonstrated in the failure for image-based models to incorporate text data. Each of the CNN models fails to incorporate text data in their models for predicting gender classification, detecting sarcasm, profiling, and predicting social media popularity \cite{Hassner2015, Poria2016, Segalin2017, Gelli2015}. Image-based social media models fail to incorporate text-data in their methodology.

\subsection{Applied Model for Advertising and Forecasting User Behavior}

This research applies its methodology to a use case of predicting user behavior in response to advertising. We feel this use case is relevant to the Marketing Science Journal because of it demonstrates a method for improving models of user behavior on social media in response to advertising. Such topics might be utilized to provide advertisers with a competitive advantage, or alternatively used in future research to improve marketing models on social media. This paper provides marketing science about user behavior with regard to advertisements on social media.

Advertisers are most concerned with social media metrics, especially those that promote engagement \cite{Tiago2014}, which include click-through rate (CTR), brand awareness, and word-of-mouth buzz. Advertisers associate these with advertisement return on investment (ROI), which is known as the Holy Grail of social media \cite{Fisher2009}. However, advertisers calculate ROI, which often includes an increase in user interaction \cite{Romero2011, Schacht2015}. This study successfully models user engagement, which is of great interest to advertisers and social media platforms.  

Advertisers want to impact future sales from the untapped market \cite{Guo2020}. Their goals include creating brand stickiness, improving user relationship quality, creating unique visitors, increasing average time per visit to their website, get repeated visitors, and increase visit frequency \cite{Bhat2002}. There are many ways to improve advertisement campaign performance, such as influencing both its content and content type \cite{Imsa2020}. However, neither of these provides a direct forecast of the advertisement's performance. Given the cost of showing ads, quicker feedback mechanisms that can predict advertisement performance is useful in curating content and publishing on the platform with a great degree of confidence concerning the advertisement's performance \cite{Hu2016}. Therefore, this study is helpful in that it provides improved mechanisms for forecasting advertisement performance on social media.

Forecasting user response to advertisements is important because advertisers view social media as a method for creating both tangible and intangible firm value that improves business performance \cite{Authors2013}. Tangible benefits include a decreased time needed for users to make a buying decision \cite{Authors2013}. Intangible benefits include how advertisements influence buyer decisions \cite{Authors2013}. In addition, with better forecasting, advertisers can improve planning their sales cycles and projected revenue \cite{Imsa2020}. This paper provides details on improved user behavior forecasting, which is beneficial to advertising revenue. 

Social media serves as a platform where brands can create and maintain an online presence \cite{Greenwood2016}. Social media can create tangible value that improves business performance \cite{Authors2013}. The desire is that tangible user engagements result in faster user conversions \cite{Authors2013}. Social media can serve as a platform for influencing their target audiences and increase their bottom-lines.

\subsection{Research Summary}

Our research provides a method for combining text and image data for modeling user behavior on social media. We demonstrate the successful implementation of a model combining image and text data and demonstrate its improved performance over single-data type models. The chosen method makes use of an ensemble model whose input is a text-based NN and image-based CNN. The combined model outperforms the text and image models when predicting user click, share, comment, and comment sentiment. The results of the sixteen machine learning models are provided in the results section and the discussion provides insights concerning the model's performance and its application to advertising social media data. Future social media studies should adopt the combined model methodology when modeling user behavior on social media.

The remaining paper consists of five sections. The related works will cover existing studies that model user behavior on social media and will delineate studies relying on text data or image data. We also include methodologies for processing text and image data within the related works section. The methodology section will describe the creation of the sixteen machine learning models, four text-only, four image-only, and eight combined models with different architectures. The result section outlines the result of the combined model, juxtaposed with text-only and image-only models. The discussion delineates why the combined model produces an improved performance. The conclusion and future work outline ways future research can adopt these methods to better model user behavior on social media.

\section{Related Work}

\subsubsection{Text-based Social Media Models}
Text models exist to predict user interaction on Facebook \cite{8029313}. The predicted user metrics include page likes, shares, and comment counts from this data. The analysis categorizes all posts into engagement categories, e.g., low, medium, and high.  The Neural Network trains with on the text and time data. The model can accurately predict for lower user engagement but fails to predict for higher levels of engagement. The study's sample size was 100k posts and did not incorporate images or comment text in its predictions. Nevertheless, the study found that text data can predict limited levels of user engagement.

A CTR study focuses on predictions based on user interests \cite{Li2015}. The study is essential because it models the likelihood of user behavior based on user interests with advertiser data.  The study performs its prediction by modeling the Twitter feed and the click rates for each type of user interest. As a result, the study successfully predicted user click-through rates based on how well user interests coincide with the advertisement's content.

Research exists that to measure user sentiment using either text or image data. Text data is useful for opinion mining \cite{Liu2012}, where opinion mining uses keywords as sentiment indicators. Fortunately, existing sentiment lexicons are available for predicting sentence sentiment \cite{Georgiou2015}. In contrast, there is research that detects image sentiment by clustering images \cite{Wang2015}. Methods exist that use either text or image data to predict user sentiment. However, there are no cases of using a combination of image and text data to predict user sentiment on social media.

\subsubsection{Image-based Social Media Models}
Many studies use Convolutional Neural Networks (CNN) for image analysis. The use cases are varied, and include: age and gender classification \cite{Hassner2015}; image polarity \cite{Poria2016}; sarcasm detection \cite{Poria2016}; and image popularity classification \cite{Khosla2014}. Existing research has produced visual sentiment classifiers with CNNs \cite{Segalin2017,Xu2014} to identify stress within social media images, \cite{Lin2014}, use supervised CNNs to performed social profiling to identify personality traits \cite{Segalin2017}, perform sentiment analyses and estimated social media popularity with CNNs \cite{Gelli2015}, use images to predict which types of images are popular on social media \cite{Gelli2015}, and predict which posts will receive the most clicks \cite{Khosla2014}. CNNs are frequently used in combination with images on social media for understanding user behavior.

\subsection{Industry need for Modeling User Behavior}

Companies calculate social media revenue return on investment (ROI) via their advertisement performance on the platform \cite{Fisher2009}. Therefore, ROI is the Holy Grail of social media \cite{Fisher2009}. When asked which social media metrics marketing managers care about most, they replied with brand awareness, word-of-mouth buzz, customer satisfaction, user-generated content, and web analytics \cite{Tiago2014}. However, ROI is difficult to track \cite{Schacht2015}. Most companies are unable to get revenue or cost savings from social media \cite{Romero2011}. Instead, ROI is measured via user consumption \cite{Schacht2015}. The study performed a cross-platform analysis of ROI on Facebook, Twitter, and Foursquare. Schacht proved that tweets could predict rising Foursquare check-ins.

Users visit social media sites to gain information \cite{Fisher2009}—for example, 34\% of participants post products about opinions on blogs. Moreover, traffic to blogs keeps increasing 50\% alone that year, compared to 17\% at CNN, MSNBC, and the New York Times. 70\% of consumers visit social media sites for information. 49\% of the 70\% buy based on social media content. 36\% of participants better rate companies with blogs. 60\% of users pass along social media data to other users. Persons use social media to learn and gain opinions about products and brands.

\subsubsection{Research Questions}
\b{Research Question:} Can a combination of text and image data better predict user engagement on social media using machine learning?

\quad{When predicting user engagement on social media, models trained using both text and image data outperform either text or image models.}

The question explores the existing gap in using a machine learning architecture that digests both image and text data to predict user engagement. Such an architecture might include text-based NN, CNNs, and popular models like decision trees. The predicted user engagement consists of the count of likes, comments, shares, and comment sentiment. Models that predict numbers use regression and mean-squared error (MSE) as their loss function. This research explores whether model architectures that combine text and image better produce a model with a lower loss than their text and image counterparts. 

\b{Research Question:} Why are images better than text data for predicting user comment sentiment using machine learning?

\quad{Models trained on image data better predict user comment sentiment than models only trained on text data.}

Existing studies use images in CNN models to predict visual sentiment \cite{Sengalin2017
